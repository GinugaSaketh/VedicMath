\documentclass{article}
\usepackage{geometry}
\usepackage{amsmath}
\usepackage{hyperref}
\usepackage{graphicx}
\title{\LARGE{Vedic Multiplication Method } }
\date{March 6, 2017}
\author{Ginuga Saketh}


\geometry{
	a4paper,
		total={210mm,297mm},
		left=30mm,
		right=30mm,
		top=15mm,
		bottom=15mm,
}

\begin{document}
\maketitle

\section{\Large{Introduction}}
\hfill In general while multiplying, we multiply each digit of multiplier to multiplicand and add it to the product after shifting appropriately.
But, here we describe an easier method for multiplying two-digit numbers which was originally described in \href{https://en.wikipedia.org/wiki/Atharvaveda}{Atharvaveda}.
In this method there are two possible cases,which are discussed in the following sections.

\section{\Large{Case 1: }\large{Same first digit and sum of last digits is 10}}
\subsection{\large{Illustration}}
\hfill Consider that first(tens' place) digit of both numbers is same
and the sum of last(ones' place) two digits is equal to 10.
Let's take an example of multiplying two nuumbers $66$ and $64$ which is shown in the following figure.
\hfill \break
%\includegraphics[width = 40mm, height = 40mm]{case1.pdf}
\subsection{\large{Explanation}}
\hspace{3mm} Let \textit{a} be tens' place digit in both the numbers.
Let \textit{b} and \textit{c} be the ones' place digits such that $b + c = 10$. 
So, here we can obtain the first two digits of the resultant product by multiplying \textit{a} and \textit{a+1}.
The last two digits can be obtained by multiplying \textit{b} and \textit{c}. We can unserstand this like multiplying $a*10+b$ with $a*10+c$ we get first two places
from the first two places of $a*10*(a*10+10)$ i.e. $a*(a+1)$ and last two places are obtained from $b*c$
\subsection{\large{Analysis}}
\hspace{3mm} For analysing this method we compare it with the normal multiplication we usually perform to show that the vedic multiplication is much better and faster for this case. 
\hfill \break
\begin{tabular}{|c|c|c|c|}
\hline
Method  &  Multiplication Steps  &  Addition Steps  &  Total steps \footnote{We assume that multiplication and addition of two digits have equal effort.} \\
	\hline
	Vedic Multiplication  &  2  &  1  & 3\\
	\hline
	Normal Multiplication  & 4  &  4  & 8\\
	\hline
	\end{tabular}
	\hfill \break
	\hspace{3mm} Hence, as we can see from the above\footnote{hello} 
\end{tabluar}
\subsection{Algebraic Analysis}
\hspace{3mm} As mentioned in the \ref{ssec:exp1} Explanation, we consider two numbers with same first digit a and last two digits b and c.
\begin{align}
	Sum\ of\ last\ digits = b+c = 10 \label{eq:cond1}
\end{align}
\begin{align}
	Product &= (a*10 + b) * (a*10 +c) \\
			&= 100*a*a + 10*a*c + 10*a*b +b*c \\
			&= 10*a*(a*10+10) + b*c  (From \eqref{eq:cond1})\\
	Product &= 100*a*(a + 1) + b*c \label{eq:1}
\end{align}
\hspace{3mm} Hence, from Equation \eqref{eq:1}, we can see that $a*(a+1)$ gives the first digits and $b*c$ gives last two digits.

\end{document}
